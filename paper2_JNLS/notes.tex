\documentclass[11pt]{article}
%\documentclass[preprint, 10pt]{elsarticle}

% PACKAGES
\usepackage{graphicx, amsmath, amssymb, amsfonts, mathtools, mathrsfs, color}
\usepackage{comment, enumerate, tabularx}
\usepackage{natbib, hyperref, url}
\usepackage[margin=1in]{geometry}
%\usepackage[justification=RaggedRight]{caption}

%----------------------------------------------------------------------------%
%% LATEX DEFINITIONS
%----------------------------------------------------------------------------%
% Basic editing
\newcommand{\tocite}{{\color{blue}(to cite)}}
\newcommand{\vsp}[1]{\vspace{#1 pc} \noindent}
\newcommand{\np}{\newpage \noindent}
\newcommand{\seclabel}[1]{\vsp{1} {\bf#1:}}

% Derivatives
\newcommand{\pd}[2]    { \frac{\partial #1} {\partial #2} }
\newcommand{\ppd}[2]  { \frac{\partial^2 #1}{{\partial #2}^2} }
\newcommand{\pdi}[2] { {\partial_#2} #1 }
\newcommand{\td}[2] { \frac{d #1} { d #2 } }
\newcommand{\grad}{\nabla}
\newcommand {\Lap} {\grad^2}
% Vectors and operators
\newcommand{\bvec}[1]{\ensuremath{\boldsymbol{#1}}}
\newcommand{\abs}[1]{\left| #1 \right|}
\newcommand{\norm}[1]{\left\| #1 \right\|}
\newcommand{\mean}[1]{\left< #1 \right>}
\newcommand{\eps}{\varepsilon}
\newcommand{\dx}{\, dx}
% Basic physical parameters and scales
\newcommand{\freqp}{f_p}
\newcommand{\etastd}{\eta_{\text std}}
% Parameters that change crossing the ADC
\newcommand{\depth}{d}
\newcommand{\dup}{\depth_{-}}
\newcommand{\ddn}{\depth_{+}}
\newcommand{\dupdn}{\depth_{\pm}}
\newcommand{\lam}{\lambda}
\newcommand{\lamup}{\lam_{-}}
\newcommand{\lamdn}{\lam_{+}}
\newcommand{\lamupdn}{\lam_{\pm}}
\newcommand{\lamfac}{N}
\newcommand{\drat}{\mathcal{D}}
\newcommand{\dratdn}{\drat_+}
\newcommand{\dratupdn}{\drat_{\pm}}
\newcommand{\DD}{\dratdn}
% Statistical quantities
\newcommand{\En}{\mathcal{E}}
\newcommand{\Mo}{\mathcal{M}}
\newcommand{\skw}{\text{skew}}
\newcommand{\skwdn}{\skw_+}
\newcommand{\var}{\text{var}}
\newcommand{\varup}{\var_-}
\newcommand{\vardn}{\var_+}
% Dimensionless parameters
\newcommand{\ampscale}{\mathcal{A}}
\newcommand{\lengthscale}{\mathcal{L}}
\newcommand{\timescale}{\mathcal{T}}
\newcommand{\epsup}{\eps_0}
\newcommand{\delup}{\delta_0}
% Hamiltonian stuff
\newcommand{\uhat}{\hat{u}}
\newcommand{\sympJ}{\mathcal{J}}
\newcommand{\vard}[2]{\frac{\delta #1}{\delta #2}}
\newcommand{\Ham}{\mathcal{H}}
\newcommand{\Hup}{\Ham^{-}}
\newcommand{\Hdn}{\Ham^{+}}
\newcommand{\Hupdn}{\Ham^{\pm}}
\newcommand{\Hthree}{\Ham_{3}}
\newcommand{\Htwo}{\Ham_{2}}
%\newcommand{\Fcnl}{\mathcal{F}}
% Truncated stuff
\newcommand{\Proj}{\mathcal{P}_{\Lambda}}
\newcommand{\uL}{u_{\Lambda}}
\newcommand{\HLupdn}{\Ham_{\Lambda}^{\pm}}
\newcommand{\SympL}{\sympJ_{\Lambda}}
% Gibbs and theta
\newcommand{\Gibbs}{\mathcal{G}}
\newcommand{\Gup}{\Gibbs^{-}}
\newcommand{\Gdn}{\Gibbs^{+}}
\newcommand{\Gupdn}{\Gibbs^{\pm}}
\newcommand{\thup}{\theta^{-}}
\newcommand{\thdn}{\theta^{+}}
\newcommand{\thupdn}{\theta^{\pm}}
\newcommand{\meanup}[1]{\mean{#1}_{-}}
\newcommand{\meandn}[1]{\mean{#1}_{+}}
\newcommand{\meanupdn}[1]{\mean{#1}_{\pm}}
\newcommand{\meanz}[1]{\mean{#1}_0}

% Experiments
\newcommand{\omavg}{\omega_0}
\newcommand{\omsig}{\sigma_{\omega}}
%----------------------------------------------------------------------------%

%----------------------------------------------%
%% TITLE
%----------------------------------------------%
\begin{document}

\title{Notes on JNLS paper}

\author{
M.~N.~J.~Moore\thanks{Florida State University} }
\maketitle


%----------------------------------------------------------------------------%
% Transfer function
%----------------------------------------------------------------------------%
\section{Notes on the transfer function}

Consider the transfer function that maps the upstream inverse temperature to the downstream one
\begin{equation}
\thdn = F(\thup)
\end{equation}
This transfer function results from the matching condition
\begin{align}
\label{statmatch}
&\meanup{\Hdn} = \meandn{\Hdn}
\end{align}
The Hamiltonian is given by
\begin{align}
\label{Hamiltonian}
&\Hupdn = C_2 \dratupdn^{1/2} \, \Htwo - C_3 \dratupdn^{-3/2} \, \Hthree \\
&C_3 = \frac{3}{2} \pi^{1/2} \epsup \delup^{-1} \, , \quad
C_2 = \frac{2 \pi^2 \delup}{3 \lamfac^2} 
\end{align}
where
\begin{align}
\label{H3H2}
&\Hthree = \frac{1}{6} \int_{-\pi}^{\pi} u^3 \dx	\\
&\Htwo = \frac{1}{2} \int_{-\pi}^{\pi} u_x^2 \dx =
2 \pi \sum_{k=1}^{\Lambda} \abs{k}^ 2\abs{\uhat_k}^2
\end{align}

\subsection{Scaling to make the transfer function relatively independent of $\Lambda$}
{\bf Goal:} Ideally, we would like the transfer function $F$ to be approximately independent of $\Lambda$, or at least not exceedingly sensitive to $\Lambda$. This is the only way that we could reasonably match the values of $\theta$ to the experiment with some meaning.

\seclabel{Issue with scaling of $\Htwo$}
I believe the heart of the matter is that $\mean{\Htwo} \sim \Lambda^2$, which can make the matching condition and thus the transfer function too sensitive to $\Lambda$. As a demonstration, consider the uniform distribution, i.e.~Gibbs measure with zero inverse temperature, and the corresponding mean $\mean{}_0$. With this as the invariant measure, one can consider the equipartition of energy as a heuristic
\begin{align}
&\En[\uL] = 2 \pi \sum_{k=1}^{\Lambda} \abs{\uhat_k}^2 = 1 \\
&\abs{\uhat_k}^2 \approx \frac{1}{2 \pi \Lambda}	 \qquad \mbox{for equipartition}
\end{align}
Then the expected value of $\Htwo$ is
\begin{equation}
\mean{\Htwo}_0 = 2 \pi \sum_{k=1}^{\Lambda} \abs{k}^ 2 \mean{\abs{\uhat_k}^2}_0 \approx \frac{1}{3} \Lambda^2
\end{equation}
We have used the identity
\begin{equation}
\sum_{k=1}^{n} \abs{k}^ 2 = \frac{1}{6} n(n+1)(2n+1) \approx n^3/3
\end{equation}
% Like the Gauss sum.
% Source: https://brilliant.org/wiki/sum-of-n-n2-or-n3/

\seclabel{Resolution, scale $C_2$ appropriately}
Recall the definition
\begin{equation}
C_2 = \frac{2 \pi^2 \delup}{3 \lamfac^2} 
\end{equation}
In particular, if we let $\lamfac \sim \Lambda$, then we will have
\begin{equation}
\mean{C_2 \Htwo}_0 \sim \Lambda^0 
\end{equation}
which removes the extreme sensitivity issue. More specifically, lets let $\Lambda = a \lamfac$ where $a = $ 1 or 2. Then we have
\begin{equation}
\mean{C_2 \Htwo}_0 \approx \frac{2}{9} \pi^2 a^2 \delup^2
\end{equation}
which is not sensitive to the truncation wavenumber $\Lambda$.

\seclabel{Tests}
I ran some tests and they seem to support my analysis that we should scale $
\lamfac$ with $\Lambda$. The best looking results came from $\lamfac = \Lambda/2$ and $\lamfac = \Lambda/3$, where the later is integer division.



\subsection{Approximating the transfer function}
Consider the matching condition \eqref{statmatch}. The upstream mean is
\begin{align}
&\meanup{\Hdn} = \frac{\int \exp\left(-\thup \Hup\right) \Hdn d\mu}{ \int \exp\left(-\thup \Hup \right) d\mu}
\end{align}
where $d \mu$ is the uniform measure on the surface $\En=1$. Similarly, we have the downstream mean
\begin{align}
&\meandn{\Hdn} = \frac{\int \exp\left(-\thdn \Hdn\right) \Hdn d\mu}{ \int \exp\left(-\thdn \Hdn \right) d\mu}
\end{align}
In both cases, we Taylor expand $\exp(-\theta \Ham) \sim 1 - \theta \Ham$ for small $\theta$. Enforcing the matching condition gives
\begin{equation}
\thup \int \Hup (\Hdn -1 ) d\mu = \thdn \int \Hdn (\Hdn - 1) d\mu
\end{equation}
Therefore an approximation to the Transfer function is 
\begin{equation}
\thdn = F(\thup) \approx m \thup
\end{equation}
where
\begin{equation}
m = \frac{  \int \Hup (\Hdn -1 ) d\mu }{ \int \Hdn (\Hdn - 1) d\mu }
\end{equation}
Using the definition of the Hamiltonian gives
\begin{equation}
m = \frac{ C_2^2 \DD^{1/2} \meanz{\Htwo^2} 
	- C_2 C_3(\DD^{-3/2} + \DD^{1/2})  \meanz{\Hthree \Htwo} 
	+ C_3^2 \DD^{-3/2} \meanz{\Hthree^2} - C_2 \meanz{\Htwo}}
	{ C_2^2 \DD \meanz{\Htwo^2} 
	- 2 C_2 C_3 \DD^{-1} \meanz{\Hthree \Htwo} 
	+ C_3^2 \DD^{-3} \meanz{\Hthree^2} - C_2 \DD^{1/2} \meanz{\Htwo}}
\end{equation}

I believe that by symmetry $\meanz{\Hthree \Htwo} = 0$
\\ TO CONTINUE

\section{Testing the computation of the Hamiltonian in the Matlab DNS and Julia Sampling}

\seclabel{Goal} We need some test problems for which we can compute the Hamiltonian exactly, to make sure it is done correctly in the Matlab DNS and Julia sampling codes.

\seclabel{Test examples} \\
1. $u = b \left( (a+\cos{x})^2 - (a^2+1/2) \right)$ \\
i. The above satisfies $\Mo[u] = 0$ (due to the choice of constant $(a^2 + 1/2)$. \\
ii. The choice $b = \sqrt{ \frac{8}{\pi(1+16 a^2)}}$ satisfies $\En[u] = 1$ \\
Wolfram alpha gives that
\begin{align}
&\Htwo[u] = \frac{1}{2} \int_{-\pi}^{\pi} u_x^2 \dx = 1 + \frac{3}{1+16 a^2} \\
&\Hthree[u] = \frac{1}{6} \int_{-\pi}^{\pi} u^3 \dx = 8 \sqrt{2} \pi a^2 \left( 16 \pi a^2 + \pi \right)^{-3/2}
\end{align}
$\bullet$ I tested the Matlab DNS code and it works! I tested $\Htwo$ and $\Hthree$ separately. Each gave a relative error on the order of machine precision.

\end{document}

